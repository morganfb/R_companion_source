
\section{Inference for a Single Proportion}
Looking at the example presented in Section 8.1 of \textit{OI Biostat}, a dataset contains 80 cancer patients at the Dana Farber Cancer Institute (DFCI) who survived at least 5 years and 40 patients who did not.  \textsf{R} can be used to "create" this dataset and to infer information about the population.  The initial statistics calculated are the mean, the standard deviation, and the sample proportion, $\hat{p}$, which can be seen in the code below.  

\begin{knitrout}
\definecolor{shadecolor}{rgb}{0.969, 0.969, 0.969}\color{fgcolor}\begin{kframe}
\begin{alltt}
\hlstd{x} \hlkwb{=} \hlkwd{c}\hlstd{(}\hlkwd{rep}\hlstd{(}\hlnum{1}\hlstd{,}\hlnum{80}\hlstd{),}\hlkwd{rep}\hlstd{(}\hlnum{0}\hlstd{,}\hlnum{40}\hlstd{))}
\hlstd{x.bar} \hlkwb{=} \hlkwd{mean}\hlstd{(x)}
\hlstd{std.dev} \hlkwb{=} \hlkwd{sd}\hlstd{(x)}
\hlstd{p.hat} \hlkwb{=} \hlkwd{sum}\hlstd{(x)}\hlopt{/}\hlkwd{length}\hlstd{(x)}
\hlstd{p.hat}
\end{alltt}
\begin{verbatim}
## [1] 0.6666667
\end{verbatim}
\end{kframe}
\end{knitrout}

\subsection{Inference Using the Normal Approximation}
Following the criteria laid out in \textit{OI Biostat}, the DFCI data can be approximated with a normal distribution.  The method for making inferences about the data based on this assumption is laid out as follows.  

\subsubsection{Confidence Intervals}
The same method as discussed in previous chapters can be used to build a confidence interval, following the form 
\[ \hat{p} \pm z^* \cdot SE_{\hat{p}}\]

An example of this worked out is from \textit{OI Biostat} Example 8.3.  
\begin{knitrout}
\definecolor{shadecolor}{rgb}{0.969, 0.969, 0.969}\color{fgcolor}\begin{kframe}
\begin{alltt}
\hlcom{## Example 8.3}
\hlstd{SE.p.hat} \hlkwb{=} \hlkwd{sqrt}\hlstd{(p.hat}\hlopt{*}\hlstd{(}\hlnum{1}\hlopt{-}\hlstd{p.hat)}\hlopt{/}\hlkwd{length}\hlstd{(x))}
\hlkwd{c}\hlstd{(p.hat} \hlopt{-} \hlnum{1.96}\hlopt{*}\hlstd{SE.p.hat, p.hat} \hlopt{+}  \hlnum{1.96}\hlopt{*}\hlstd{SE.p.hat)}
\end{alltt}
\begin{verbatim}
## [1] 0.5823217 0.7510116
\end{verbatim}
\end{kframe}
\end{knitrout}

\subsubsection{Hypothesis Testing}
Again, the procedures that have been performed before can be used to do a hypothesis test on a proportion.  The $z$-statistic would be 
\[ z = \frac{\hat{p}-p_0}{SE_{p_0}}\]

Example 8.5 from \textit{OI Biostat} is worked out below.  Note that there is a slight difference in the $z$ statistic value because this version does not use an approximation of $\hat{p}$ to 0.67.  
\begin{knitrout}
\definecolor{shadecolor}{rgb}{0.969, 0.969, 0.969}\color{fgcolor}\begin{kframe}
\begin{alltt}
\hlcom{## Example 8.5}
\hlstd{p.0} \hlkwb{=} \hlnum{0.6}
\hlstd{SE.p.0} \hlkwb{=} \hlkwd{sqrt}\hlstd{(p.0}\hlopt{*}\hlstd{(}\hlnum{1}\hlopt{-}\hlstd{p.0)}\hlopt{/}\hlkwd{length}\hlstd{(x))}
\hlstd{z} \hlkwb{=} \hlstd{(p.hat} \hlopt{-} \hlstd{p.0)}\hlopt{/}\hlstd{SE.p.0}
\hlstd{z}
\end{alltt}
\begin{verbatim}
## [1] 1.490712
\end{verbatim}
\begin{alltt}
\hlcom{## get the p-value (note that it is a two-sided test)}
\hlnum{2}\hlopt{*}\hlkwd{pnorm}\hlstd{(z,} \hlkwc{lower.tail} \hlstd{=} \hlnum{FALSE}\hlstd{)}
\end{alltt}
\begin{verbatim}
## [1] 0.1360371
\end{verbatim}
\end{kframe}
\end{knitrout}

\section{Inference for the Difference of Two Proportions}


\section{Inference for Two or More Groups}


\section{Chi-Square} 


\section{Randomization Test}
