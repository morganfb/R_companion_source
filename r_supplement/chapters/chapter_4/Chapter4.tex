\vspace{0.5cm} 

\section{Taking a Sample of a Dataset}
The main text talks through the importance of using a sample of data as a reflection of the population as a whole.  The two steps for doing this are as follows
\begin{enumerate}
\item Of the total number of datapoints you have (equal to the length of the dataset), select a random sampling of them, of size $n$.  This is done with the \textit{sample()} command below.  $x$ is a list of the row indeces, and \textit{sample} returns a random list of values in that list of length $n$. 
\item Taking the output of the \textit{sample} command, pair these indeces with the dataset to get the corresponding rows (or columns).  
\end{enumerate}

%% Figure out what this seed should be to recreate the book examples perfectly!
\begin{knitrout}
\definecolor{shadecolor}{rgb}{0.969, 0.969, 0.969}\color{fgcolor}\begin{kframe}
\begin{alltt}
\hlkwd{set.seed}\hlstd{(}\hlnum{111}\hlstd{)}
\hlcom{# Step 1: Take a sample of the row indeces }
\hlstd{indeces} \hlkwb{=} \hlkwd{sample}\hlstd{(}\hlkwc{x} \hlstd{= (}\hlnum{1}\hlopt{:}\hlkwd{nrow}\hlstd{(yrbss)),} \hlkwc{size} \hlstd{=} \hlnum{100}\hlstd{,} \hlkwc{replace} \hlstd{=} \hlnum{FALSE}\hlstd{)}
\end{alltt}


{\ttfamily\noindent\bfseries\color{errorcolor}{\#\# Error in nrow(yrbss): object 'yrbss' not found}}\begin{alltt}
\hlcom{# Step 2: Pull those corresponding individuals from the dataset }
\hlstd{samp} \hlkwb{=} \hlstd{yrbss[indeces,]}
\end{alltt}


{\ttfamily\noindent\bfseries\color{errorcolor}{\#\# Error in eval(expr, envir, enclos): object 'yrbss' not found}}\end{kframe}
\end{knitrout}

Standard histograms plots of some of the variables on this dataset can be plotted.  
\begin{knitrout}
\definecolor{shadecolor}{rgb}{0.969, 0.969, 0.969}\color{fgcolor}\begin{kframe}
\begin{alltt}
\hlcom{## Figure 4.4 }
\hlkwd{par}\hlstd{(}\hlkwc{mfrow}\hlstd{=}\hlkwd{c}\hlstd{(}\hlnum{2}\hlstd{,}\hlnum{2}\hlstd{))}
\hlkwd{hist}\hlstd{(samp}\hlopt{$}\hlstd{height,} \hlkwc{main} \hlstd{=} \hlstr{"Height"}\hlstd{)}
\end{alltt}


{\ttfamily\noindent\bfseries\color{errorcolor}{\#\# Error in hist(samp\$height, main = "{}Height"{}): object 'samp' not found}}\begin{alltt}
\hlkwd{hist}\hlstd{(samp}\hlopt{$}\hlstd{weight,} \hlkwc{main} \hlstd{=} \hlstr{"Weight"}\hlstd{)}
\end{alltt}


{\ttfamily\noindent\bfseries\color{errorcolor}{\#\# Error in hist(samp\$weight, main = "{}Weight"{}): object 'samp' not found}}\begin{alltt}
\hlkwd{hist}\hlstd{(samp}\hlopt{$}\hlstd{physically.active.7d,} \hlkwc{main} \hlstd{=} \hlstr{"Physical Activity in Past Week"}\hlstd{)}
\end{alltt}


{\ttfamily\noindent\bfseries\color{errorcolor}{\#\# Error in hist(samp\$physically.active.7d, main = "{}Physical Activity in Past Week"{}): object 'samp' not found}}\begin{alltt}
\hlkwd{hist}\hlstd{(samp}\hlopt{$}\hlstd{strength.training.7d,} \hlkwc{main} \hlstd{=} \hlstr{"Lifting Weights in Past Week"}\hlstd{)}
\end{alltt}


{\ttfamily\noindent\bfseries\color{errorcolor}{\#\# Error in hist(samp\$strength.training.7d, main = "{}Lifting Weights in Past Week"{}): object 'samp' not found}}\end{kframe}
\end{knitrout}

\section{Variability in Estimates}
The sample parameters can be calculated as well as the population parameters.  
\begin{knitrout}
\definecolor{shadecolor}{rgb}{0.969, 0.969, 0.969}\color{fgcolor}\begin{kframe}
\begin{alltt}
\hlkwd{mean}\hlstd{(samp}\hlopt{$}\hlstd{physically.active.7d,} \hlkwc{na.rm} \hlstd{=} \hlnum{TRUE}\hlstd{)}
\end{alltt}


{\ttfamily\noindent\bfseries\color{errorcolor}{\#\# Error in mean(samp\$physically.active.7d, na.rm = TRUE): object 'samp' not found}}\begin{alltt}
\hlkwd{mean}\hlstd{(yrbss}\hlopt{$}\hlstd{physically.active.7d,} \hlkwc{na.rm} \hlstd{=} \hlnum{TRUE}\hlstd{)}
\end{alltt}


{\ttfamily\noindent\bfseries\color{errorcolor}{\#\# Error in mean(yrbss\$physically.active.7d, na.rm = TRUE): object 'yrbss' not found}}\begin{alltt}
\hlkwd{median}\hlstd{(samp}\hlopt{$}\hlstd{physically.active.7d,} \hlkwc{na.rm} \hlstd{=} \hlnum{TRUE}\hlstd{)}
\end{alltt}


{\ttfamily\noindent\bfseries\color{errorcolor}{\#\# Error in median(samp\$physically.active.7d, na.rm = TRUE): object 'samp' not found}}\begin{alltt}
\hlkwd{median}\hlstd{(yrbss}\hlopt{$}\hlstd{physically.active.7d,} \hlkwc{na.rm} \hlstd{=} \hlnum{TRUE}\hlstd{)}
\end{alltt}


{\ttfamily\noindent\bfseries\color{errorcolor}{\#\# Error in median(yrbss\$physically.active.7d, na.rm = TRUE): object 'yrbss' not found}}\begin{alltt}
\hlkwd{sd}\hlstd{(samp}\hlopt{$}\hlstd{physically.active.7d,} \hlkwc{na.rm} \hlstd{=} \hlnum{TRUE}\hlstd{)}
\end{alltt}


{\ttfamily\noindent\bfseries\color{errorcolor}{\#\# Error in is.data.frame(x): object 'samp' not found}}\begin{alltt}
\hlkwd{sd}\hlstd{(yrbss}\hlopt{$}\hlstd{physically.active.7d,} \hlkwc{na.rm} \hlstd{=} \hlnum{TRUE}\hlstd{)}
\end{alltt}


{\ttfamily\noindent\bfseries\color{errorcolor}{\#\# Error in is.data.frame(x): object 'yrbss' not found}}\end{kframe}
\end{knitrout}

\subsection{Sampling Distribution for the Mean}
The concept of a sampling distribution highlights the fact that sampling is a random process, and every sample is likely to be quite different than any other.  For this reason, sampling distributions are created, which represent the accumulated information of a large number of random samples.  

The following steps can be used to create a \textbf{sampling distribution of the sample mean}.  
\begin{knitrout}
\definecolor{shadecolor}{rgb}{0.969, 0.969, 0.969}\color{fgcolor}\begin{kframe}
\begin{alltt}
\hlcom{## Figure 4.7}
\hlcom{# Step 1: Create an empty list to put values in }
\hlstd{means} \hlkwb{=} \hlkwd{rep}\hlstd{(}\hlnum{NA}\hlstd{,} \hlnum{1000}\hlstd{)}

\hlcom{# Step 2: Use a "for" loop to collect 1000 samples }
\hlkwa{for}\hlstd{(ii} \hlkwa{in} \hlnum{1}\hlopt{:}\hlnum{1000}\hlstd{)\{}
  \hlcom{# Step 2a: Get the random sample }
  \hlstd{indeces} \hlkwb{=} \hlkwd{sample}\hlstd{(}\hlkwc{x} \hlstd{= (}\hlnum{1}\hlopt{:}\hlkwd{nrow}\hlstd{(yrbss)),} \hlkwc{size} \hlstd{=} \hlnum{100}\hlstd{,} \hlkwc{replace} \hlstd{=} \hlnum{FALSE}\hlstd{)}
  \hlstd{samp} \hlkwb{=} \hlstd{yrbss[indeces,]}
  \hlcom{# Step 2b: Take the mean of the sample and store it }
  \hlstd{means[ii]} \hlkwb{=} \hlkwd{mean}\hlstd{(samp}\hlopt{$}\hlstd{physically.active.7d,} \hlkwc{na.rm} \hlstd{=} \hlnum{TRUE}\hlstd{)}
\hlstd{\}}
\end{alltt}


{\ttfamily\noindent\bfseries\color{errorcolor}{\#\# Error in nrow(yrbss): object 'yrbss' not found}}\begin{alltt}
\hlcom{# Step 3: Plot your results }
\hlkwd{par}\hlstd{(}\hlkwc{mfrow} \hlstd{=} \hlkwd{c}\hlstd{(}\hlnum{1}\hlstd{,}\hlnum{2}\hlstd{))}
\hlkwd{hist}\hlstd{(means,} \hlkwc{breaks} \hlstd{=} \hlnum{30}\hlstd{)}
\end{alltt}


{\ttfamily\noindent\bfseries\color{errorcolor}{\#\# Error in hist.default(means, breaks = 30): 'x' must be numeric}}\begin{alltt}
\hlkwd{qqnorm}\hlstd{(means)}
\end{alltt}


{\ttfamily\noindent\bfseries\color{errorcolor}{\#\# Error in qqnorm.default(means): y is empty or has only NAs}}\end{kframe}
\end{knitrout}

The more samples are taken, the more accurately will the distribution be depicted.  Figure 4.8 demonstrates this below 
\begin{knitrout}
\definecolor{shadecolor}{rgb}{0.969, 0.969, 0.969}\color{fgcolor}\begin{kframe}
\begin{alltt}
\hlstd{means} \hlkwb{=} \hlkwd{rep}\hlstd{(}\hlnum{NA}\hlstd{,} \hlnum{100000}\hlstd{)}
\hlkwa{for}\hlstd{(ii} \hlkwa{in} \hlnum{1}\hlopt{:}\hlnum{100000}\hlstd{)\{}
  \hlstd{indeces} \hlkwb{=} \hlkwd{sample}\hlstd{(}\hlkwc{x} \hlstd{= (}\hlnum{1}\hlopt{:}\hlkwd{nrow}\hlstd{(yrbss)),} \hlkwc{size} \hlstd{=} \hlnum{100}\hlstd{,} \hlkwc{replace} \hlstd{=} \hlnum{FALSE}\hlstd{)}
  \hlstd{samp} \hlkwb{=} \hlstd{yrbss[indeces,]}
  \hlstd{means[ii]} \hlkwb{=} \hlkwd{mean}\hlstd{(samp}\hlopt{$}\hlstd{physically.active.7d,} \hlkwc{na.rm} \hlstd{=} \hlnum{TRUE}\hlstd{)}
\hlstd{\}}

\hlkwd{par}\hlstd{(}\hlkwc{mfrow} \hlstd{=} \hlkwd{c}\hlstd{(}\hlnum{1}\hlstd{,}\hlnum{2}\hlstd{))}
\hlkwd{hist}\hlstd{(means,} \hlkwc{breaks} \hlstd{=} \hlnum{30}\hlstd{)}
\hlkwd{qqnorm}\hlstd{(means)}
\end{alltt}
\end{kframe}
\end{knitrout}

